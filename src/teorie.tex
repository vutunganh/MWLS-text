\chapter{Teorie}

\section{Definice a značení}

Reálná čísla standardně značíme zdvojeným $\reals$. Symbolem $\hat n$ značíme množinu čísel $\{1, 2, \ldots, n\}$. Vektory $\vec{v} \in \reals^n$ značíme písmenem s šipkou. Matice $\mx{A} \in \reals^{n, m}$ značíme zdvojeným písmenem. Normu značíme $\norm{\cdot}_k$. Pokud vypouštíme index $k$ u normy, pak myslíme Euklidovskou normu $\norm{\cdot}_2$. Euklidovskou vzdálenost mezi dvojicí vektorů $\vec{u}$ a $\vec{v}$ tedy značíme $\norm{\vec{u} - \vec{v}}$. V této kapitole občas použijeme porovnání mezi vektory $\vec{u} < \vec{v}$. Vektor $\vec{u}$ je menší než vektor $\vec{v}$, pokud $\forall i \in \hat{n}: \vec{u_i} < \vec{v_i}$.

% \section{Metoda nejmenších čtverců}

% % TODO: citace studijní text BI-VMM
% Nechť je dána sada dat $\{(x_i, y_i)\}_{i = 1}^n$ a chceme najít lineární kombinaci daných funkcí $f_1, f_2, \ldots, f_m$, tj. funkci
% $$
% f = \sum_{i = 1}^m c_if_i
% $$
% tak, aby funkční hodnoty funkce $f$ v bodech $x_i$ co nejlépe odpovídaly hodnotám $y_i$ pro každé $i \in \hat n$. Úkolem je určit neznáme koeficienty lineární kombinace $c_1, c_2, \ldots, c_m$. Předpokládejme, že $m \leq n$, typicky je množštví dat $n$ mnohem větší než počet funkcí $m$.

% % TODO: volba f

% Metoda nejmenších čtverců spočívá v myšlence minimalizovat kvadrát celkové chyby mezi $y_i$ a $f(x_i)$. Přesněji řečeno, hledáme konstanty $c_1, c_2, \ldots, c_m$ tak, aby hodnota
% $$
% F(c) = \sum_{i = 1}^n (y_i - f(x_i))^2 = \sum_{i = 1}^n (y_i - (Ac)_i)^2 = \|y - Ac\|^2_2
% $$
% byla co nejmenší. Zde jsme označili
% $$
% (Ac)_i = \sum_{j = 1}^{m} f_j(x_i)c_j.
% $$
% Matice $A \in \reals^{n, m}$ je dána hodnotami jednotlivých funkcí $f_1, f_2, \ldots, f_m$ v bodech $x_1, x_2, \ldots, x_n$, kde \label[def-A]$A_{i, j} := f_j(x_i)$.

% \subsection{Jiná definice, ještě se musí vybrat lepší z nich}
% Nechť je dáno $N$ bodů $x_i \in \reals{}^d$, kde $i \in \{1, \ldots, N\}$ a $N$ skalárních hodnot $f_i \in \reals{}$. Úkolem je nalézt funkci $f$, která minimalizuje chybu ve smyslu nejmenších čtverců s chybovou funkcí $J = \sum_i \| f(x_i) - f_i \|^2$. Řešená minimalizační úloha má následující předpis
% \label[ls-statement]
% $$ \min_{f \in \prod^{d}_{m}}\sum_i \| f(x_i) - f_i \|^2$$
% kde $f$ je funkce z $\prod^{d}_{m}$, prostoru polynomů $m$ proměnných stupně $d$.
% Funkci $f$ lze zapsat jako
% $$ f(x_i) = b(x_i) c, $$
% kde $b(x) = (b_1(x), \ldots, b_k(x))^T$ je bazický vektor prostoru polynomů a $c = (c_1, \ldots, c_k)^T$ je vektor neznámých koeficientů, které chceme minimalizovat v \ref[ls-statement].
% Například pro prostor $\prod^{2}_{2}$ je bazický vektor $(1, x, y, x^2, xy, y^2)$.

% Minimalizace $A$

\section{Problém hledání nejbližších sousedů}

\label{defi:nns}
\begin{defi}
  Nechť je dána sada vektorů $S := \{\vec{x}_i\}_{i = 1}^n$, kde každé $\vec{x}_i \in \reals^d$. Naším úkolem nalést pro vektor $\vec{q} \in \reals^d$ a $\delta \in \reals$ podmnožinu $M \subseteq S$ takovou, že pro všechny vektory $\vec{v} \in M$ platí, že $\norm{\vec{v} - \vec{q}} \leq \delta$ a pro všechny vektory $\vec{w} \in S \setminus M$ platí, že $\norm{\vec{w} - \vec{q}} > \delta$ a tedy $\vec{w} \not\in M$.
\end{defi}

Tuto úlohu lze řešit naivně algoritmem \ref{algo:naive-nns}, který pro každý vektor $\vec{u} \in S$ zjistí vzdálenost $\norm{\vec{u} - \vec{q}}$ a pokud je menší než $\delta$, pak $u$ přidá do výsledné množiny. Za předpokladu, že přidávání do výsledné množiny a počítání vzdálenosti mezi 2 vektory zvládneme v čase $O(1)$, tento algoritmus běží v čase $O(|S|)$ a zabírá $O(|S|)$ paměti.

\begin{algorithm}[h!]
  \caption{Naivní řešení problému hledání nejbližších sousedů}
  \label{algo:naive-nns}
  \begin{algorithmic}
    \REQUIRE $S \subset \reals^n$, $\vec{q} \in \reals^n$, $\delta \in \reals$
    \ENSURE $M \subseteq S$
    \STATE $M \leftarrow \emptyset$
    \FORALL{$\vec{u} \in S$}
        \IF{$\norm{\vec{u} - \vec{q}} \leq \delta$}
          \STATE $M \leftarrow M \cup \vec{u}$
        \ENDIF
    \ENDFOR
    \RETURN $M$
  \end{algorithmic}
\end{algorithm}

V této sekci se budeme zabývat 2 řešeními tohoto problému, které byly použity v implementační části práce. Cílem je zejména snížit časovou složitost při hledání nejbližších sousedů. Budeme se zabývat pouze exaktními metodami.

\subsection{Cell linked list}

Naivní způsob, který prohledá všech $n$, by mohl být časově příliš pomalý. Cell linked list rozdělí prostor $\reals^d$ na $d$-rozměrné krychle s délkou hrany $\epsilon$, které se volí blízko k $\delta$. Nyní stačí najít hyperkrychli, do které patří $q$ a prohledat $\lceil \delta / \epsilon \rceil^d$ sousedních hyperkrychlí.

\subsection{K-d strom}

K--d strom je binárním vyhledávacím stromem, jehož klíči v jednotlivých vrcholech jsou body z $\reals{}^k$. Pro účely této sekce zavádíme následující značení: nechť $v = (v_0, ..., v_{k - 1})  \in \reals^k$ je klíčem vrcholu $P$, pak jednotlivé složky značíme $K_0(P), K_1(P), \ldots, K_{k - 1}(P)$. Ukazatele vrcholu $P$ na svého levého a pravého potomka značíme $L(P)$, resp. $R(P)$. Diskriminátor vrcholu $P$ značíme $D(P)$.

Každému vrcholu $P$ k--d stromu $T$ přiřazujeme tzv. diskriminátor, jenž je přirozeným číslem o $0$ do $k - 1$. Nechť $j := D(P)$, pak 

V této části popíšeme pouze ty operace, které jsou v práci použity. Jmenovitě vypouštíme operace přidávání prvku, mazání prvku a hledání sousedů v rozsahu.

