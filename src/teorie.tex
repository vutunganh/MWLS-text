\chap Teorie

\sec Definice a značení

Symbolem $\hat n$ značíme množinu čísel $\{1, 2, \ldots, n\}$. Vektory $\vector{v} \in \reals^n$ značíme písmenem s šipkou. Matice $\matrix{A} \in \reals^{n, m}$ značíme zdvojeným písmenem.

\sec Metoda nejmenších čtverců

% TODO: citace studijní text BI-VMM
Nechť je dána sada dat $\{(\vector{x_i}, y_i)\}_{i = 1}^n$ a chceme najít lineární kombinaci daných funkcí $f_1, f_2, \ldots, f_m$, tj. funkci
$$
f = \sum_{i = 1}^m c_if_i
$$
tak, aby funkční hodnoty funkce $f$ v bodech $x_i$ co nejlépe odpovídaly hodnotám $y_i$ pro každé $i \in \hat n$. Úkolem je určit neznáme koeficienty lineární kombinace $c_1, c_2, \ldots, c_m$. Předpokládejme, že $m \leq n$, typicky je množštví dat $n$ mnohem větší než počet funkcí $m$.

% TODO: volba f

Metoda nejmenších čtverců spočívá v myšlence minimalizovat kvadrát celkové chyby mezi $y_i$ a $f(x_i)$. Přesněji řečeno, hledáme konstanty $c_1, c_2, \ldots, c_m$ tak, aby hodnota
$$
F(\vector{c}) = \sum_{i = 1}^n (y_i - f(x_i))^2 = \sum_{i = 1}^n (y_i - (\matrix{A}\vector{c})_i)^2 = \|\vector{y} - \matrix{A}\vector{c}\|^2_2
$$
byla co nejmenší. Zde jsme označili
$$
(\matrix{A}\vector{c})_i = \sum_{j = 1}^{m} f_j(\vector{x}_i)c_j.
$$

\secc Jiná definice, ještě se musí vybrat lepší z nich
Nechť je dáno $N$ bodů $\vector{x_i} \in {\bbchar R}^d$, kde $i \in \{1, \ldots, N\}$ a $N$ skalárních hodnot $f_i \in \bbchar R$. Úkolem je nalézt funkci $f$, která minimalizuje chybu ve smyslu nejmenších čtverců s chybovou funkcí $J = \sum_i \| f(\vector{x_i}) - f_i \|^2$. Řešená minimalizační úloha má následující předpis
\label[ls-statement]
$$ \min_{f \in \prod^{d}_{m}}\sum_i \| f(\vector{x_i}) - f_i \|^2, \eqmark $$
kde $f$ je funkce z $\prod^{d}_{m}$, prostoru polynomů $m$ proměnných stupně $d$.
Funkci $f$ lze zapsat jako
$$ f(\vector{x_i}) = \vector{b}(\vector{x_i}) \vector{c}, $$
kde $\vector{b}(\vector{x}) = (b_1(\vector{x}), \ldots, b_k(\vector{x}))^T$ je bazický vektor prostoru polynomů a $\vector{c} = (c_1, \ldots, c_k)^T$ je vektor neznámých koeficientů, které chceme minimalizovat v \ref[ls-statement].
Například pro prostor $\prod^{2}_{2}$ je bazický vektor $(1, x, y, x^2, xy, y^2)$.

{\bf Řešení. } Minimalizace $\matrix{A}$

\sec Cell linked list

Cell linked list[\rcite[computer_simulation_of_liquids],~s.~149--152] je datová struktura, kterou lze řešit problém hledání nejbližších sousedů. Nechť je dána sada bodů $S := \{\vector{x_i}\}_{i = 1}^n$, kde každé $x_i \in \reals^d$. Chceme pro libovolný dotazovaný bod $q \in \reals^d$ a $\delta \in \reals$ nalézt podmnožinu $S' \subseteq S$ takovou, že pro libovolný bod $\vector{p'} \in S'$ platí, že $\|q - p'\| \leq \delta$ a pro každý bod $p' \in S \setminus S'$ platí, že $\|q - p'\| > \delta$.

Naivní způsob, který prohledá všech $n$, by mohl být časově příliš pomalý. Cell linked list rozdělí prostor $\reals^d$ na $d$-rozměrné krychle s délkou hrany $\epsilon$, které se volí blízko k $\delta$. Nyní stačí najít hyperkrychli, do které patří $q$ a prohledat $\lceil {\delta \over \epsilon} \rceil^d$ sousedních hyperkrychlí.


