\chapter{Teorie}

\section{Definice a značení}

Symbolem $\hat n$ značíme množinu čísel $\{1, 2, \ldots, n\}$. Vektory $v \in \reals^n$ značíme písmenem s šipkou. Matice $A \in \reals^{n, m}$ značíme zdvojeným písmenem.

\section{Metoda nejmenších čtverců}

% TODO: citace studijní text BI-VMM
Nechť je dána sada dat $\{(x_i, y_i)\}_{i = 1}^n$ a chceme najít lineární kombinaci daných funkcí $f_1, f_2, \ldots, f_m$, tj. funkci
$$
f = \sum_{i = 1}^m c_if_i
$$
tak, aby funkční hodnoty funkce $f$ v bodech $x_i$ co nejlépe odpovídaly hodnotám $y_i$ pro každé $i \in \hat n$. Úkolem je určit neznáme koeficienty lineární kombinace $c_1, c_2, \ldots, c_m$. Předpokládejme, že $m \leq n$, typicky je množštví dat $n$ mnohem větší než počet funkcí $m$.

% TODO: volba f

Metoda nejmenších čtverců spočívá v myšlence minimalizovat kvadrát celkové chyby mezi $y_i$ a $f(x_i)$. Přesněji řečeno, hledáme konstanty $c_1, c_2, \ldots, c_m$ tak, aby hodnota
$$
F(c) = \sum_{i = 1}^n (y_i - f(x_i))^2 = \sum_{i = 1}^n (y_i - (Ac)_i)^2 = \|y - Ac\|^2_2
$$
byla co nejmenší. Zde jsme označili
$$
(Ac)_i = \sum_{j = 1}^{m} f_j(x_i)c_j.
$$
Matice $A \in \reals^{n, m}$ je dána hodnotami jednotlivých funkcí $f_1, f_2, \ldots, f_m$ v bodech $x_1, x_2, \ldots, x_n$, kde \label[def-A]$A_{i, j} := f_j(x_i)$.

\subsection{Jiná definice, ještě se musí vybrat lepší z nich}
Nechť je dáno $N$ bodů $x_i \in \reals{}^d$, kde $i \in \{1, \ldots, N\}$ a $N$ skalárních hodnot $f_i \in \reals{}$. Úkolem je nalézt funkci $f$, která minimalizuje chybu ve smyslu nejmenších čtverců s chybovou funkcí $J = \sum_i \| f(x_i) - f_i \|^2$. Řešená minimalizační úloha má následující předpis
\label[ls-statement]
$$ \min_{f \in \prod^{d}_{m}}\sum_i \| f(x_i) - f_i \|^2$$
kde $f$ je funkce z $\prod^{d}_{m}$, prostoru polynomů $m$ proměnných stupně $d$.
Funkci $f$ lze zapsat jako
$$ f(x_i) = b(x_i) c, $$
kde $b(x) = (b_1(x), \ldots, b_k(x))^T$ je bazický vektor prostoru polynomů a $c = (c_1, \ldots, c_k)^T$ je vektor neznámých koeficientů, které chceme minimalizovat v \ref[ls-statement].
Například pro prostor $\prod^{2}_{2}$ je bazický vektor $(1, x, y, x^2, xy, y^2)$.

Minimalizace $A$

\section{Problém hledání nejbližších sousedů}

\label[def-nns]
Nechť je dána sada bodů $S := \{x_i\}_{i = 1}^n$, kde každé $x_i \in \reals^d$. Naším úkolem je pro libovolný bod $q \in \reals^d$ a vzdálenost $\delta \in \reals$ nalézt podmnožinu $M \subseteq S$ takovou, že pro každý bod $v \in M$ platí, že $\|q - v\| \leq \delta$ a pro každý bod $u \not\in M$ platí, že $\|q - v\| > \delta$.

Tuto úlohu lze řešit naivně tak, že pro každý bod $v \in S$ zjistíme vzdálenost $\|v - q\|$ a pokud je menší nebo rovna $\delta$, pak přidáme $v$ do výsledné množiny.

V této kapitole se budeme zabývat 2 řešeními tohoto problému. 

\subsubsection{Cell linked list}

Cell linked list\cite[,~s.~149--152]{computer_simulation_of_liquids} je datová struktura, kterou lze řešit problém hledání nejbližších sousedů. Nechť je dána sada bodů $S := \{x_i\}_{i = 1}^n$, kde každé $x_i \in \reals^d$. Chceme pro libovolný dotazovaný bod $q \in \reals^d$ a $\delta \in \reals$ nalézt podmnožinu $S' \subseteq S$ takovou, že pro libovolný bod $p' \in S'$ platí, že $\|q - p'\| \leq \delta$ a pro každý bod $p' \in S \setminus S'$ platí, že $\|q - p'\| > \delta$.

Naivní způsob, který prohledá všech $n$, by mohl být časově příliš pomalý. Cell linked list rozdělí prostor $\reals^d$ na $d$-rozměrné krychle s délkou hrany $\epsilon$, které se volí blízko k $\delta$. Nyní stačí najít hyperkrychli, do které patří $q$ a prohledat $\lceil {\delta \over \epsilon} \rceil^d$ sousedních hyperkrychlí.

\subsubsection{K-d strom}

K--d strom je binárním vyhledávacím stromem, jehož klíči v jednotlivých vrcholech jsou body z $\reals{}^k$. Pro účely této sekce zavádíme následující značení: nechť $v = (v_0, ..., v_{k - 1})  \in \reals^k$ je klíčem vrcholu $P$, pak jednotlivé složky značíme $K_0(P), K_1(P), \ldots, K_{k - 1}(P)$. Ukazatele vrcholu $P$ na svého levého a pravého potomka značíme $L(P)$, resp. $R(P)$. Diskriminátor vrcholu $P$ značíme $D(P)$.

Každému vrcholu $P$ k--d stromu $T$ přiřazujeme tzv. diskriminátor, jenž je přirozeným číslem o $0$ do $k - 1$. Nechť $j := D(P)$, pak 

V této části popíšeme pouze ty operace, které jsou v práci použity. Jmenovitě vypouštíme operace přidávání prvku, mazání prvku a hledání sousedů v rozsahu.

