\chap Úvod

Předpokládejme, že je zadána sada dat s jejími výstupními hodnotami $S := \{(x_i, y_i)\}_{i = 1}^n$. Naším cílem je nalézt funkci $f$, která co nejlépe popisuje vztah mezi vstupy a výstupy dat $S$.

Nejstarší metodou pro řešení této úlohy je {\it metoda nejmenších čtverců} z 19. století, která se připisuje pánům Adrien Marie Legendre a Carl Fridrich Gauss. Adrien Marie Legendre v roce 1805 metodu popsal ve svém díle ,,Nouvelles Méthodes pour la Détermination des Orbites des Com\`etes'', ve kterém ji aplikoval k aproximaci tvaru Země.[\rcite[history_of_statistics],~s.~12--14].

V práci se budeme věnovat vylepšení, které se nazývá {\it metoda pohyblivých vážených nejmenších čtverců}. Kromě přesnějších výsledků dokáže metoda například interpolovat nad vstupními daty.

Častým problémem při programování numerických metod a podobně výkonných programů je ,,problém dvou jazyků'' -- prototypování probíhá ve dynamických vysokoúrovňových jazycích jako jsou například Matlab, R nebo Python a výsledná implementace vznikne přepsáním prototypu do výkonnějších, staticky typovaných jazyků, kde častými volbami jsou Fortran nebo C. Autoři jazyku Julia si dávají za cíl vytvořit dynamicky typovaný jazyk, který bude výkonem schopen konkurovat nízkoúrovňovému Fortranu nebo C.\cite[doi:10.1137/141000671]

Hlavním výstupem práce by měla být implementace metody pohyblivých vážených nejmenších čtverců v programovacím jazyce Julia.

V teoretické části si zformulujeme problém, který se dá řešit metodou nejmenších čtverců. Představíme metodu pohyblivých vážených nejmenších čtverců. Dále budeme pokračovat rešerší existujicích implementací metody a rešerší datových struktur, jímiž lze řešit stěžejní podproblém metody pohyblivých vážených nejmenších čtverců a to problém hledání nejbližších bodů. V poslední části si popíšeme návrh a naši implementaci v jazyce Julia, jež je hlavním výstupem této práce.

