\chap Úvod

V této práci se budeme zabývat metodou nejmenších čtverců a některými jejími vylepšeními.

V první části si zformulujeme problém, který se dá řešit metodou nejmenších čtverců. Představíme si {\it metodu pohyblivých vážených nejmenších čtverců}.

Dále budeme pokračovat rešerší existujicích implementací metody a rešerší datových struktur, jímiž lze řešit stěžejní podproblém metody pohyblivých vážených nejmenších čtverců a to problém hledání nejbližších bodů.

V poslední části si popíšeme naši implementaci v jazyce Julia, jež je hlavním výstupem této práce. Také zinterpretujeme výsledky měření a testování.

\sec Cíle

Cílem práce by měla být hotová implementace metody nejmenších čtverců, rešerše vhodných datových struktur použitých v této metodě a jejich případná implementace a testování a měření výsledného programu.
