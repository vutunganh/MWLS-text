%%%%%%%%%%%%%
% Example thesis (CTU FIT) based on CTUstyle 
% use: pdfcsplain <FIT-ZP.tex>
% write this document as UTF-8
%%%%%%%%%%%%%
% CTUstyle: see more at http://petr.olsak.net/ctustyle.html
%%%%%%%%%%%%%

% This example created by Ondrej Guth <ondrej.guth@fit.cvut.cz> 2014
% based on CTUstyle by Petr Olsak

\input ctustyle2
\overfullrule=0pt

\worktype [B/CZ]  % (thesis type {B,M,D})/(language {CZ,EN})
\faculty {F8}
\department {Katedra teoretické informatiky}
\title {Metoda pohyblivých vážených nejmenších čtverců v Julia}
\author {Tung Anh Vu}
\date {Leden 2018}
\supervisor{Ing. Tomáš Kalvoda, Ph.D.}
\thanks {Děkuji svému vedoucímu Ing. Tomáši Kalvodovi, Ph.D., za cenné rady, připomínky a pravidelné konzultace, které mi byly vždy cennou zpětnou vazbou.}
\abstractEN {An implementation of moving weighted least squares method was created.}
\abstractCZ {V této práci se implementovala metoda nejmenších pohyblivých čtverců.}
\declaration {
   Prohlašuji, že jsem předloženou práci vypracoval(a) samostatně a že jsem
   uvedl(a) veškeré použité informační zdroje v~souladu s~Metodickým
   pokynem o~etické přípravě vysokoškolských závěrečných prací.

   Beru na vědomí, že se na moji práci vztahují práva a povinnosti
   vyplývající ze zákona č.~121/2000~Sb., autorského zákona, ve znění
   pozdějších předpisů, zejména skutečnost, že České vysoké učení technické
   v~Praze má právo na uzavření licenční smlouvy o~užití této práce jako
   školního díla podle §~60 odst.~1 autorského zákona.

   V Praze dne 1. 5. 2018 % !!! Attention, you have to change this item.
   \signature % makes dots
}

\makefront

\input uvod.tex

% \sec Prerekvizity a další informace

% Tento dokument slouží jako příklad, jak vytvořit závěrečnou práci s použitím
% CTUstyle, který vytvořil Petr Olšák. Následující zdroje
% informací obsahují makra, která můžete použít při psaní své práce.

% Více informací o CTUstyle najdete na \url{http://petr.olsak.net/ctustyle.html} -- návod k pou{ž}ití, povinné {č}ásti, tipy. Doporu{č}ujeme p{ř}e{č}íst.

% Styl je založený na sadě maker OPmac, kterou najdete (včetně dokumentace) na
% \url{http://petr.olsak.net/opmac.html} -- doporučujeme prohlédnout.

% \sec Použití na FIT

% Kromě samotného textu práce nastavte základní informace o práci (volání před
% příkazem makefront). U svého jména uvádějte tituly, které máte v
% okamžiku odevzdání. Text prohlášení je nutné doslova převzít z
% následující kapitoly (to je opravdu nutné).


% \label[sec:prohlaseni]
% \chap Texty prohlášení


% Vyberte si jeden z následujících textů \uv{prohlášení} (strana iii) a ten
% (doslova a bez jakýchkoli úprav) použijte (příkaz "\declaration").
% Jakoukoli modifikaci musí předem schválit studijní oddělení fakulty.

% \sec Minimální

%    Prohlašuji, že jsem předloženou práci vypracoval(a) samostatně a že jsem
%    uvedl(a) veškeré použité informační zdroje v~souladu s~Metodickým pokynem
%    o~etické přípravě vysokoškolských závěrečných prací.

%    Beru na vědomí, že se na moji práci vztahují práva a povinnosti vyplývající
%    ze zákona č.~121/2000~Sb., autorského zákona, ve znění pozdějších předpisů,
%    zejména skutečnost, že České vysoké učení technické v~Praze má právo na
%    uzavření licenční smlouvy o~užití této práce jako školního díla podle
%    §~60 odst.~1 autorského zákona.

% \sec Udělení práv škole

%    Prohlašuji, že jsem předloženou práci vypracoval(a) samostatně a že jsem
%    uvedl(a) veškeré použité informační zdroje v~souladu s~Metodickým pokynem
%    o~etické přípravě vysokoškolských závěrečných prací.

%    Beru na vědomí, že se na moji práci vztahují práva a povinnosti vyplývající
%    ze zákona č.~121/2000~Sb., autorského zákona, ve znění pozdějších předpisů.
%    V~souladu s~ust. §~46 odst.~6 tohoto zákona tímto uděluji nevýhradní
%    oprávnění (licenci) k~užití této mojí práce, a to včetně všech počítačových
%    programů, jež jsou její součástí či přílohou a veškeré jejich dokumentace
%    (dále souhrnně jen \uv{Dílo}), a to všem osobám, které si přejí Dílo užít.
%    Tyto osoby jsou oprávněny Dílo užít jakýmkoli způsobem, který nesnižuje
%    hodnotu Díla, avšak pouze k~nevýdělečným účelům. Toto oprávnění je časově,
%    teritoriálně i množstevně neomezené.
		
% \sec Licenční smlouva se školou

%    Prohlašuji, že jsem předloženou práci vypracoval(a) samostatně a že jsem
%    uvedl(a) veškeré použité informační zdroje v~souladu s~Metodickým pokynem
%    o~etické přípravě vysokoškolských závěrečných prací.

%    Beru na vědomí, že se na moji práci vztahují práva a povinnosti vyplývající
%    ze zákona č.~121/2000~Sb., autorského zákona, ve znění pozdějších předpisů.
%    Dále prohlašuji, že jsem s~Českým vysokým učením technickým v~Praze
%    uzavřel licenční smlouvu o~užití této práce jako školního díla podle §~60
%    odst.~1 autorského zákona. Tato skutečnost nemá vliv na ust. §~47b zákona
%    č.~111/1998~Sb., o~vysokých školách, ve znění pozdějších předpisů.

% \sec Otevřená licence

%    Prohlašuji, že jsem předloženou práci vypracoval(a) samostatně a že jsem
%    uvedl(a) veškeré použité informační zdroje v~souladu s~Metodickým pokynem
%    o~etické přípravě vysokoškolských závěrečných prací.

%    Beru na vědomí, že se na moji práci vztahují práva a povinnosti vyplývající
%    ze zákona č.~121/2000~Sb., autorského zákona, ve znění pozdějších předpisů.
%    V~souladu s~ust. §~46 odst.~6 tohoto zákona tímto uděluji nevýhradní
%    oprávnění (licenci) k~užití této mojí práce, a to včetně všech počítačových
%    programů, jež jsou její součástí či přílohou, a veškeré jejich dokumentace
%    (dále souhrnně jen \uv{Dílo}), a to všem osobám, které si přejí Dílo užít.
%    Tyto osoby jsou oprávněny Dílo užít jakýmkoli způsobem, který nesnižuje
%    hodnotu Díla, a za jakýmkoli účelem (včetně užití k~výdělečným účelům).
%    Toto oprávnění je časově, teritoriálně i~množstevně neomezené. Každá osoba,
%    která využije výše uvedenou licenci, se však zavazuje udělit ke každému
%    dílu, které vznikne (byť jen zčásti) na základě Díla, úpravou Díla,
%    spojením Díla s~jiným dílem, zařazením Díla do díla souborného či
%    zpracováním Díla (včetně překladu), licenci alespoň ve výše uvedeném
%    rozsahu a zároveň zpřístupnit zdrojový kód takového díla alespoň
%    srovnatelným způsobem a ve srovnatelném rozsahu, jako je zpřístupněn
%    zdrojový kód Díla.

% \sec Oprávnění k užití

%    Prohlašuji, že jsem předloženou práci vypracoval(a) samostatně a že jsem
%    uvedl(a) veškeré použité informační zdroje v~souladu s~Metodickým pokynem
%    o~etické přípravě vysokoškolských závěrečných prací.

%    Beru na vědomí, že se na moji práci vztahují práva a povinnosti vyplývající
%    ze zákona č.~121/2000~Sb., autorského zákona, ve znění pozdějších předpisů.
%    V~souladu s~ust. §~46 odst.~6 tohoto zákona tímto uděluji nevýhradní
%    oprávnění (licenci) k~užití této mojí práce, a to včetně všech počítačových
%    programů, jež jsou její součástí či přílohou a veškeré jejich dokumentace
%    (dále souhrnně jen \uv{Dílo}), a to všem osobám, které si přejí Dílo užít.
%    Tyto osoby jsou oprávněny Dílo užít jakýmkoli způsobem, který nesnižuje
%    hodnotu Díla a za jakýmkoli účelem (včetně užití k~výdělečným účelům).
%    Toto oprávnění je časově, teritoriálně i množstevně neomezené.

% \sec Po dohodě s ČVUT

%    Prohlašuji, že jsem předloženou práci vypracoval(a) samostatně a že jsem
%    uvedl(a) veškeré použité informační zdroje v~souladu s~Metodickým pokynem
%    o~etické přípravě vysokoškolských závěrečných prací.

%    Beru na vědomí, že se na moji práci vztahují práva a povinnosti vyplývající
%    ze zákona č.~121/2000~Sb., autorského zákona, ve znění pozdějších předpisů.
%    Dále prohlašuji, že jsem s~Českým vysokým učením technickým v~Praze
%    uzavřel dohodu, na základě níž se ČVUT vzdalo práva na uzavření licenční
%    smlouvy o~užití této práce jako školního díla podle §~60 odst.~1
%    autorského zákona. Tato skutečnost nemá vliv na ust. §~47b zákona
%    č.~111/1998~Sb., o~vysokých školách, ve znění pozdějších předpisů.

\bye
