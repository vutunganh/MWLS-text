\documentclass[thesis=B,czech]{FITthesis}[2012/06/26]

\usepackage[utf8]{inputenc} % LaTeX source encoded as UTF-8
\usepackage[czech]{babel}

\usepackage[backend=biber, sorting=none]{biblatex}

\usepackage{graphicx} %graphics files inclusion

\usepackage{amsmath} %advanced maths
\usepackage{amssymb} %additional math symbols
\usepackage{amsthm}

\usepackage[chapter]{algorithm}
\usepackage{algorithmic}

\usepackage{csquotes}
\usepackage{xevlna}

\usepackage{enumerate}

\usepackage{dirtree} %directory tree visualisation

% % list of acronyms
% \usepackage[acronym,nonumberlist,toc,numberedsection=autolabel]{glossaries}
% \iflanguage{czech}{\renewcommand*{\acronymname}{Seznam pou{\v z}it{\' y}ch zkratek}}{}
% \makeglossaries

\newcommand{\tg}{\mathop{\mathrm{tg}}} %cesky tangens
\newcommand{\cotg}{\mathop{\mathrm{cotg}}} %cesky cotangens
\newcommand{\reals}{\mathbb{R}}
\newcommand{\integers}{\mathbb{N}}
\newcommand{\mx}[1]{\mathbb{#1}}
\newcommand{\norm}[1]{\|#1\|}

\addbibresource{literatura.bib}

\newtheoremstyle{newthm}%
  {4pt}         % space above
  {4pt}         % space below
  {\normalfont} % body font
  {}            % indent amount
  {\bfseries}   % theorem head font
  {:}           % punctuation after theorem head
  {.5em}        % space after theorem head
  {}            % head spec (?)

\theoremstyle{newthm}
\newtheorem{defi}{Definice}
\newtheorem{veta}[defi]{Věta}
\newtheorem{lemma}[defi]{Lemma}
\newtheorem{dusl}[defi]{Důsledek}

\renewcommand{\algorithmicrequire}{\textbf{Vstup:}}
\renewcommand{\algorithmicensure}{\textbf{Výstup:}}
\makeatletter
\renewcommand{\ALG@name}{Algoritmus}
\makeatother

\department{Katedra teoretické informatiky}
\title{Metoda pohyblivých nejmenších čtverců v Julia}
\authorGN{Tung Anh} %(křestní) jméno (jména) autora
\authorFN{Vu} %příjmení autora
\authorWithDegrees{Tung Anh Vu} %jméno autora včetně současných akademických titulů
\author{Tung Anh Vu} %jméno autora bez akademických titulů
\supervisor{Ing. Tomáš Kalvoda, Ph.D.}
\acknowledgements{Děkuji svému vedoucímu Ing. Tomáši Kalvodovi, Ph.D., za cenné rady, připomínky a pravidelné konzultace, které mi byly vždy cennou zpětnou vazbou.}
\abstractCS{V této práci se implementovala metoda nejmenších čtverců v jazyce Julia. Byl formulován problém hledání nejbližších sousedů a předvedla se 2 řešení toho problému. Také lze nalézt krátký popis programovacího jazyka Julia.}
\abstractEN{The moving weighted least squares was implemented. Nearest neighbor search problem was formulated and 2 solutions were presented. Also a short description of Julia programming language can be found.}
\placeForDeclarationOfAuthenticity{V~Praze}
\declarationOfAuthenticityOption{4} %volba Prohlášení (číslo 1-6)
\keywordsCS{metoda pohyblivých nejmenších čtverců, programovací jazyk Julia, hledání nejbliších sousedů, aproximace funkce}
\keywordsEN{moving weighted least squares, Julia programming language, nearest neighbor search, function approximation}
% \website{http://site.example/thesis} %volitelná URL práce, objeví se v tiráži - úplně odstraňte, nemáte-li URL práce

\begin{document}

% \newacronym{CVUT}{{\v C}VUT}{{\v C}esk{\' e} vysok{\' e} u{\v c}en{\' i} technick{\' e} v Praze}
% \newacronym{FIT}{FIT}{Fakulta informa{\v c}n{\' i}ch technologi{\' i}}

\begin{introduction}
Předpokládejme, že je zadána sada dat s jejími výstupními hodnotami $S := \{(x_i, y_i)\}_{i = 1}^n$. Naším cílem je nalézt funkci $f$, která co nejlépe popisuje vztah mezi vstupy a výstupy dat $S$.

Nejstarší metodou pro řešení této úlohy je metoda nejmenších čtverců z 19. století, která se připisuje pánovi A. M. Legendre, který v roce 1805 metodu popsal ve svém díle ,,Nouvelles Méthodes pour la Détermination des Orbites des Com\`etes'', ve kterém ji aplikoval k aproximaci tvaru Země.\cite[s.~12--14]{history_of_statistics}.

V práci se budeme věnovat vylepšení, které se nazývá metoda pohyblivých vážených nejmenších čtverců. Metoda pro každý bod $\vec{x}$ vytvoří lokální aproximaci $f_{\vec{x}}(\vec{x})$, která kromě přesnějšího výsledku dokáže například interpolovat nad vstupními daty.

Častým problémem při programování numerických metod a podobně výkonných programů je ,,problém dvou jazyků'' -- prototypování probíhá ve dynamických vysokoúrovňových jazycích jako jsou například Matlab, R nebo Python a výsledná implementace vznikne přepsáním prototypu do výkonnějších, staticky typovaných jazyků, kde častými volbami jsou Fortran nebo C. Autoři jazyku Julia si dávají za cíl vytvořit dynamicky typovaný jazyk, který bude výkonem schopen konkurovat nízkoúrovňovému Fortranu nebo C.\cite{julia}

Hlavním výstupem práce by měla být implementace metody pohyblivých vážených nejmenších čtverců v programovacím jazyce Julia.

V teoretické části si zformulujeme problém, který se dá řešit metodou nejmenších čtverců. Představíme metodu pohyblivých vážených nejmenších čtverců. Dále budeme pokračovat rešerší existujicích implementací metody a rešerší datových struktur, jímiž lze řešit stěžejní podproblém metody pohyblivých vážených nejmenších čtverců a to problém hledání nejbližších bodů. V poslední části si popíšeme návrh a naši implementaci v jazyce Julia, jež je hlavním výstupem této práce.
\end{introduction}

\chapter{Teorie}

\section{Definice a značení}

Reálná čísla standardně značíme zdvojeným $\reals$. Symbolem $\hat n$ značíme množinu čísel $\{1, 2, \ldots, n\}$. Vektory $\vec{v} \in \reals^n$ značíme písmenem s šipkou. Matice $\mx{A} \in \reals^{n, m}$ značíme zdvojeným písmenem. Normu značíme $\norm{\cdot}_k$. Není-li spodní index $k$ u normy napsán, pak se myslí Euklidovská norma $\norm{\cdot}_2$. Euklidovskou vzdálenost mezi dvojicí vektorů $\vec{u}$ a $\vec{v}$ tedy značíme $\norm{\vec{u} - \vec{v}}$. V této kapitole se používá porovnání mezi vektory $\vec{u} < \vec{v}$, kde $\vec{u}, \vec{v} \in \reals^d$, které zavádíme takto: vektor $\vec{u}$ je menší než vektor $\vec{v}$, pokud $\forall i \in \hat{d}: \vec{u_i} < \vec{v_i}$, tedy je menší po složkách.

\section{Metoda nejmenších čtverců}

Nechť je dána sada dat $\{\vec{x_i}, \vec{y_i}\}_{i = 1}^k$, kde $x_i \in \reals^n$ a $y_i \in \reals^m$ a sada funkcí $f_1,\ldots,f_l$. Úkolem je nalézt funkci
\begin{equation}
  f = \sum_{i=1}^l c_if_i,
\end{equation}
s neznámými koeficienty $c_1,\ldots,c_l$, která aproximuje sadu dat co nejlépe ve smyslu nejmenších čtverců s chybovou funkcí
\begin{equation}
\label{def:ls-min}
  J(f) := \sum_{i=1}^k \norm{f(\vec{x_i}) - \vec{y_i}}^2 =
  \sum_{i=1}^k\norm{\sum_{j=1}^l c_jf_j(\vec{x_i}) - (\vec{y_i})}^2 =
  \sum_{i=1}^k\sum_{u=1}^m\sum_{j=1}^l (c_j(f_j(\vec{x_i}))_u - (\vec{y_i})_u)^2.
\end{equation}

Minimalizační úlohu \ref{def:ls-min} lze řešit pomocí nutné podmínky pro existenci lokálního extrému. Položme parciální derivace funkce $J(f)$ rovny nule pro neznáme $c_1,\ldots,c_m$. Pro obecné $v \in \hat l$ platí
\begin{equation}
  \frac{\partial}{\partial c_v}J(f) = 2\sum_{i=1}^k\sum_{u=1}^m\sum_{j=1}^l(c_j(f_j(\vec{x_i}))_u - (\vec{y_i})_u)(f_v(\vec{x_i}))_u.
\end{equation}

Položme tedy tyto derivace rovny nule a dostáváme pro jednotlivá $v \in \hat l$
\begin{multline}
\label{eq:solve-ls}
  2\sum_{i=1}^k\sum_{u=1}^m\sum_{j=1}^l (c_j(f_j(\vec{x_i}))_u - (\vec{y_i})_u)(f_v(\vec{x_i}))_u = 0\\
  \sum_{i=1}^k\sum_{u=1}^m\sum_{j=1}^l c_j(f_j(\vec{x_i}))_u (f_v(\vec{x_i}))_u = \sum_{i=1}^k\sum_{u=1}^m(\vec{y_i})_u(f_v(\vec{x_i}))_u\\
  \sum_{j=1}^l(\sum_{i=1}^k\sum_{u=1}^m(f_j(\vec{x_i}))_u (f_v(\vec{x_i}))_u)c_j = \sum_{i=1}^k\sum_{u=1}^m(\vec{y_i})_u(f_v(\vec{x_i}))_u.\\
\end{multline}
Zaveďme matici $\mx{A} \in \reals^{l,l}$, kde 
\begin{equation}
\label{eq:solve-ls-lhs}
  \mx{A}_{vj} = \sum_{i=1}^k\sum_{u=1}^m = (f_j(\vec{x_i}))_u(f_v(\vec{x_i}))_u
\end{equation}
a vektor $\vec{c} = (c_1,\ldots,c_l)$. Nyní lze levou stranu rovnice \ref{eq:solve-ls} zapsat jako $\mx{A}\vec{c}$. Dále zaveďme vektor $\vec{b} = (b_1,\ldots,b_l)$, kde pro jednotlivá $v \in \hat l$ platí
\begin{equation}
\label{eq:solve-ls-rhs}
  \vec{b}_v = \sum_{i=1}^k\sum_{u=1}^m (\vec{y_i})_u (f_v(\vec{x_i}))_u.
\end{equation}
Dosaďme rovnice \ref{eq:solve-ls-lhs} a \ref{eq:solve-ls-rhs} do \ref{eq:solve-ls} a dostáváme soustavu lineárních rovnic s vektorem neznáných $\vec{c}$
\begin{equation}
  \mx{A}\vec{c} = \vec{b}.
\end{equation}

\section{Problém hledání nejbližších sousedů}

\begin{defi}
\label{defi:nns}
  Nechť je dána sada vektorů $S := \{\vec{x}_i\}_{i = 1}^n$, kde každé $\vec{x}_i \in \reals^d$. Naším úkolem nalézt pro vektor $\vec{q} \in \reals^d$ a $\delta \in \reals$ podmnožinu $M \subseteq S$ takovou, že pro všechny vektory $\vec{v} \in M$ platí, že $\norm{\vec{v} - \vec{q}} \leq \delta$ a pro všechny vektory $\vec{w} \in S \setminus M$ platí, že $\norm{\vec{w} - \vec{q}} > \delta$ a tedy $\vec{w} \not\in M$.
\end{defi}

Tuto úlohu lze řešit naivně algoritmem \ref{algo:naive-nns}, který pro každý vektor $\vec{u} \in S$ zjistí vzdálenost $\norm{\vec{u} - \vec{q}}$ a pokud je menší než $\delta$, pak $u$ přidá do výsledné množiny. Za předpokladu, že přidávání do výsledné množiny a počítání vzdálenosti mezi 2 vektory zvládneme v čase $O(1)$, tento algoritmus běží v čase $O(|S|)$ a zabírá $O(|S|)$ paměti.

\begin{algorithm}[ht!]
  \caption{Naivní řešení problému hledání nejbližších sousedů}
  \label{algo:naive-nns}
  \begin{algorithmic}
    \REQUIRE $S \subset \reals^n$, $\vec{q} \in \reals^n$, $\delta \in \reals$
    \ENSURE $M \subseteq S$
    \STATE $M \leftarrow \emptyset$
    \FORALL{$\vec{u} \in S$}
        \IF{$\norm{\vec{u} - \vec{q}} \leq \delta$}
          \STATE $M \leftarrow M \cup \vec{u}$
        \ENDIF
    \ENDFOR
    \RETURN $M$
  \end{algorithmic}
\end{algorithm}

V této sekci se budeme zabývat 2 řešeními tohoto problému, které byly použity v implementační části práce. Cílem je zejména snížit časovou složitost při hledání nejbližších sousedů.

\subsection{Cell linked list}

\label{defi:cll}
\begin{defi}
  Nechť je dána množina vektorů $S \in \reals^d$. Datová struktura \textit{cell linked list} rozdělí prostor $V := \reals^d$ na $d$-rozměrnou rovnoměrnou ,,hyperkrychlovou'' mřížku s délkou hrany $\varepsilon$ a v každé buňce si udržuje seznam vektorů z $S$, který se v ní nachází.\cite[s.~149--152]{computer_simulation_of_liquids}
\end{defi}

Cell linked list podporuje následující operace.

\begin{table}[ht!]
  \begin{tabular}{lll}
    Operace & Komentář\\
    CllBuild & Sestavení struktury.\\
    CllCount & Spočte pro sadu dat rozměry děleného prostoru.\\
    CllSearch & Nalezení sousedů.\\
    CllRebuild & Přenos buněk mezi cell linked listy.\\
    CllAdd & Přidání vektoru do struktury.\\
    CllRemove & Odebrání vektoru ze struktury.\\
  \end{tabular}
\end{table}

Pro cell linked list $C$ zavádíme následující pomocné funkce: $\varepsilon(C)$ vrátí délku hrany mřížky, kterou je rozdělen prostor $\reals^d$. Pro přistoupení k mřížce cell linked listu $C$ používáme $G(C)$. Chceme-li navíc vybrat konkrétní buňku, která se v $C$ nachází na souřadnicích $\vec{c}$, pak používáme značení $G(C, \vec{c})$. Vektor velikosti mřížky značíme $|S(G)|$. Množina vektorů $S$ uchovávána v cell linked listu $C$ značíme $S(C)$. Data, která si cell linked list uchovává na $\vec{c}$-tých souřadnicích, značíme $S(C, \vec{c})$.
Funkce $\min(C)$ vrátí vektor $\vec{p} \in \reals^d$, který má v $i$-té složce nejmenší hodnotu $i$-té složky přes všechny vektory $S(C)$. Tento postup je zachycen algoritmem \ref{algo:min-cll}. Funkce $\max(C)$ funguje obdobně, akorát na řádku \ref{algo:min-cll:infty} se do proměnné $\vec{p}$ nastaví vektor $(-\infty, \ldots, \infty)$ a na řádku \ref{algo:min-cll:minmax} budeme hledat maximum místo minima. Tento postup lze analogicky aplikovat na množinu vektorů a pro množinu vektorů $W \in \reals^d$ budeme operaci značit podobně $\min(W)$ resp. $\max(W)$.

\begin{algorithm}[!h]
  \caption{Funkce \textit{min} cell linked listu}
  \label{algo:min-cll}
  \begin{algorithmic}[1]
    \REQUIRE cell linked list $C$
    \ENSURE $\vec{p} \in \reals^d$
    \STATE $\vec{p} \leftarrow (\infty, \ldots, \infty)^T$ \label{algo:min-cll:infty}
      \FORALL{$\vec{u} \in S(G)$}
        \FORALL{$i \in {1, \ldots, d}$}
          \STATE $\vec{p_i} \leftarrow \min(\vec{p_i}, \vec{u_i})$ \label{algo:min-cll:minmax}
        \ENDFOR
    \ENDFOR
    \RETURN $\vec{p}$
  \end{algorithmic}
\end{algorithm}

Souřadnice vektoru $\vec{u}$ v cell linked listu $C$ budeme značit $I(C, \vec{u})$ a zavádíme jako $I(C, \vec{u}) := ((1 / \varepsilon(C)) \vec{u} + \min(C))$.

\textit{CllCount} pro sadu vektorů $S$ spočte velikost co nejmenší mřižky s délkou hrany $\varepsilon$ tak, aby bylo možné v mřížce $G$ zaindexovat každý vektor množiny $S$. Nejprve se spočte minimální a maximální hodnota přes všechny složky všech vektorů množiny $S$ a tyto vektory se uloží do. TODO: tohle se osemetne formuluje. Hodnotu \textit{CllCount} pro cell linked list $C$ spočteme vztahem $\mathit{CllCount}(C) := \lceil\max(C)/\varepsilon\rceil - \lfloor\min(C)/\varepsilon\rfloor + (1, \ldots, 1)^T$. Uvažme nejprve, že chceme do struktury ukládat pouze vektory s nezápornými složkami. Pak zjevně stačí, aby mřížka obsahovala $\lceil\max(C)/\varepsilon\rceil$ buněk. V případě, že připustíme i vektory se zápornými TODO: proof

Operace \textit{CllBuild} pro zadanou sadu vektorů $S$ sestrojí co nejmenší $d$-rozměrnou pravidelnou mřížku takovou, že pro každý vektor z množiny $S$ půjde nalézt buňku, do které patří. Funkcí \textit{CllCount} se spočte velikost výsledné mřížky a v každé buňce se vytvoří z počátku prázdný spojový seznam vektorů z $\reals^d$. Následně se pro každý vektor $\vec{u} \in S$ spočte jeho souřadnice v cell linked listu a připojí se ke spojovému seznamu. Tento algoritmus je doplněn pseudokódem \ref{algo:cll-build}.

\begin{algorithm}[h!]
  \caption{Algoritmus CllBuild}
  \label{algo:cll-build}
  \begin{algorithmic}
    \REQUIRE $S \subset \reals^d$, $\varepsilon \in \reals$
    \ENSURE cell linked list $C$ obsahující množinu $S$
    \STATE $\varepsilon(C) \leftarrow \varepsilon$
    \STATE $|S(G)| \leftarrow \max(S) - \min(S) + (1, \ldots, 1)^T$
    \STATE $G(C) \leftarrow d\hbox{-rozměrné pole prázdných spojových seznamů velikosti } |S(G)|$
    \FORALL{$\vec{u} \in S$}
      \STATE insert $\vec{u}$ into $G(C, I(C, \vec{u}))$
    \ENDFOR
    \RETURN $C$
  \end{algorithmic}
\end{algorithm}

Operace \textit{CllSearch} pro daný vektor $\vec{q} \in \reals^d$ a reálnou hodnotu $\delta \in \reals$ nalezne všechny vektory $\vec{u}$ v množině $S(C)$ takové, jejichž Euklidovská vzdálenost $\|\vec{u} - \vec{q}\|$ je menší než dané $\delta$. Nebude však naivně procházet celou množinu $S(C)$. Místo toho se podívá pouze vektory v buňkách se souřadnicemi $\vec{a}$ takovými, jejichž maximová norma $\|\vec{c} - \vec{a}\|_\infty$ je menší nebo rovna $r := \lceil\delta / \varepsilon(C)\rceil$. Lze si rozmyslet, že pokud se zvolí TODO: best r

\begin{algorithm}[h!]
  \caption{Algoritmus CllSearch}
  \label{algo:cll-build}
  \begin{algorithmic}
    \REQUIRE $\delta \in \reals$, $\vec{q} \in \reals^d$, cell linked list $C$
    \ENSURE $M \subseteq S(G)$
    \STATE $r \leftarrow \lceil\delta / \varepsilon(C)\rceil$
    \STATE $c \leftarrow i(\vec{q}, C)$
    \STATE $M \leftarrow \emptyset$
    \FORALL{$\vec{m} \in \reals^d: \|\vec{m} - \vec{c}\|_{\infty} < r$}
      \FORALL{$\vec{u} \in S(\vec{m}, C): \|\vec{u} - \vec{q}\| < \delta$}
        \STATE $M \leftarrow M \cup \vec{u}$
      \ENDFOR
    \ENDFOR
    \RETURN $M$
  \end{algorithmic}
\end{algorithm}

\subsection{K-d strom}

K-d strom je binárním vyhledávacím stromem, jehož klíči v jednotlivých vrcholech jsou body z $\reals{}^k$. Pro účely této sekce zavádíme následující značení: nechť $v = (v_0, ..., v_{k - 1})  \in \reals^k$ je klíčem vrcholu $P$, pak jednotlivé složky značíme $K_0(P), K_1(P), \ldots, K_{k - 1}(P)$. Ukazatele vrcholu $P$ na svého levého a pravého potomka značíme $L(P)$, resp. $R(P)$. Diskriminátor vrcholu $P$ značíme $D(P)$.

Každému vrcholu $P$ k--d stromu $T$ přiřazujeme tzv. diskriminátor, jenž je přirozeným číslem o $0$ do $k - 1$. Nechť $j := D(P)$, pak 

V této části popíšeme pouze ty operace, které jsou v práci použity. Jmenovitě vypouštíme operace přidávání prvku, mazání prvku a hledání sousedů v rozsahu.



\chapter{Realizace}

\section{Programovací jazyk Julia}

Julia je relativně nový programovací jazyk (poprvé se objevil v roce 2012\cite{julia-first}) a proto by bylo na místě jej představit a zvýraznit některé důležité, ne zcela obvyklé vlastnosti.

Julia je dynamicky typovaný jazyk. Oproti známějším, dynamicky typovaným jazykům jako je například Python\footnote{pomineme-li typové anotace} nebo Javascript lze u proměnných, třídních proměnných a parametrů funkcí specifikovat typ. Pro tento účel je existuje postfixový operátor \texttt{::<název typu>}. Například \texttt{a = 20} vytvoří proměnnou \texttt{a} s typem \texttt{Int64} (64bitové celé číslo), zatímco \texttt{a::Int8 = 20} vytvoří proměnnou \texttt{a} s typem \texttt{Int8} (8bitové celé číslo). Specifikace typu se však nikdy explicitně nevyžaduje a vždy je možné ji vynechat.

Nicméně v některých případech je výhodnější typ uvést. U třídních proměnných je vždy lepší typ uvést z výkonnostních důvodů. Pokud typ není specifikován, pak musí být schopen udržet v sobě libovolný objekt a z tohoto důvodu se třídní typy bez uvedeného typu musí alokovat na haldě. Tento princip platí i u kontejnerů a vždy je lepší specifikovat typ ukládaných objektů.\cite{julia-performance}

Jako ve většině programovacích jazyků je i v Julii možné definovat si vlastní typy.

Složené typy jsou v Julii realizované strukturami, které téměř totožné se strukturami z jazyka C. Tyto struktury a jejich členské proměnné jsou implicitně \textit{immutable} (česky neměnné). Pokud jsou immutable členskými proměnnými složené typy (např. pole), pak lze měnit jejich obsahy, protože immutable jsou pouze reference na tyto objekty a ne objekty samotné. Pokud nechceme, aby byla deklarovaná struktura immutable, pak použijeme klíčové slovo \texttt{mutable} před deklarací struktury. Implicitní neměnnost nemusí být na škodu, poskytují mnoho nezanedbatelných výhod, např. snadnější srozumitelnost kódu a výkonnost plynoucí z toho, že neměnné objekty v některých případech není potřeba alokovat na haldu. Tohoto se využívá v naší implementaci, objekty pro sestrojení aproximace jsou vždy neměnné, ačkoliv v sobě obsahují pole se vstupními daty. Všechny struktury mají implicitní konstruktor s totožným názvem a přijímají tolik argumentů, kolik má struktura členských proměnných a v takovém pořadí, v jakém jsou deklarovány jednotlivé členské proměnné. Konstruktory si lze samozřejmě dodefinovat dle potřeby.

V Julii se immutable objekty předávají do funkcí hodnotou, zatímco mutable objekty se předávají referencí. Proměnné samy o sobě žádný typ nemají, jsou pouze ,,pojmenováním'' nějakého objektu.

Objektově orientované paradigma je v Julii realizováno poněkud netradičním způsobem. Abstraktní typy se deklarují klíčovým slovem \texttt{abstract}, nemají žádné členské proměnné a slouží pouze jako ,,vrchol'' v grafu typů Julie. Dědit se smí pouze z abstraktních typů. V Julii se dědičnost vztahuje pouze na chování objektů a nikoliv na členské proměnné.

Objekty sice nemají třídní metody, ale polymorfismus se dá aplikovat na funkce. Například funkce $foo(x::Number)$ dokáže pracovat s libovolným číselným typem, ať už se jedná o celé číslo či reálné číslo.

\section{Použité externí knihovny}

Při implementaci byly použity následující externí knihovny:
\begin{itemize}
  \item \texttt{DynamicPolynomials} a \texttt{MultivariatePolynomials} poskytují třídy pro práci s polynomy.
  \item \texttt{NearestNeighbors} poskytuje implementaci k-d stromu.
  \item \texttt{DataStructures} poskytuje spojový seznam.
  \item \texttt{Plots} umožňuje vykreslování grafů.
\end{itemize}

Při implementaci a testování se používal \texttt{Jupyter}, který v prohlížeči vytvoří interaktivní prostředí. Terminálové prostředí je nevhodné pro tvorbu rozsáhlejších programů, které se často pouštějí vícekrát, což by znamenalo ruční přepisování příkazů při každém provádění. Na druhou stranu modifikovat skripty často znamená, že i při malých změnách se musí znovu provést interpretace celého skriptu. Navíc je obtížné nechat si k jednotlivým příkazům zobrazit jejich dílčí výstupy. \texttt{Jupyter} kombinuje oba přístupy a umožňuje snadnou modifikaci skriptů (zvané notebooky), zobrazování výstupu každého příkazu a možnost ukládat si posloupnosti příkazů.

\section{Implementace cell linked listu}

TODO: Nějaká omáčka o nereálných implementacích z teoretické části sem.

V teoretické části se pro reprezentaci mřížky cell linked listu použilo nekonečně velké pole, což může být v praxi překážkou. Nechť je dána sada vektorů $S := \{\vec{x_i}\}_{i=1}^n$, kde $\vec{x_i} \in \reals^d$ a cell linked list $C$ s daným parametrem $\varepsilon(C)$. Chceme určit minimální možnou velikost mřížky $G(C)$ tak, aby v ní bylo možné uchovat všechny vektory z množiny $S$. Zaveďme značení $\min(V)$ a $\max(V)$, čímž se myslí nejmenší resp. největší vektor z nějaké množiny vektorů $V$. Je přiřozené, aby vektor $\min(S)$ měl souřadnice $(0, \ldots, 0)^T$. Proto v implementaci je modifikována funkce $I(C, \vec{u}), \vec{u} \in S$ a její výsledek je dán vztahem
\begin{equation}
  \label{cll-index}
  I_m(C, \vec{u}) = I(C, \vec{u}) - I(C, \min(S(C))) + (1, \ldots, 1).
\end{equation}
V Julii se pole indexují od 1, proto je k výsledku \ref{cll-index} přičten jednotkový vektor. Velikost mřížky, která je potřeba k uchování celé sady $S$ získáme spočtením funkce $I(C, \max(S))$.

Další implemenační problém nastal při hledání sousedů. Velikost mřížky $G(C)$ je často o mnoho větší než počet buněk obsahující vektory, které chceme prohledat. Nechť je dán vektor $\vec{q}$, ke kterému chceme nalézt všechny vektory cell linked listu $C$ vzdálené maximálně o $\delta \in \reals$. Určeme souřadnice $c = I_m(C, \vec{q})$. Místo naivní iterace přes celou mřížku $G(C)$ budeme pouze iterovat přes takové buňky $b \in G(C)$, které mají 

\section{Implementace metody pohyblivých vážených nejmenších čtverců}

Těžištěm implementace je objekt \texttt{MwlsObject}, který si uchovává potřebné vstupy pro výpočet požadovaných aproximací. Mezi jeho atributy patří:
\begin{itemize}
  \item \texttt{inputs} -- sada vstupních dat $\{\vec{x_i}\}_{i=1}^k, x_i \in \reals^n$,
  \item \texttt{outputs} -- sada výstupních dat $\{\vec{y_i}\}_{i=1}^k, y_i \in \reals^m$,
  \item \texttt{EPS} -- reálná hodnota udávající implicitní velikost ,,okolí'' pro konstrukci aproximace,
  \item \texttt{weightFunc} -- váhová funkce pro konstrukci aproximace.
\end{itemize}
Atributy \texttt{inputs} a \texttt{outputs} lze předat dvěma způsoby, buď jako dvě různá vícerozměrná pole, nebo jedním vícerozměrným polem a kladným celočíselným parametrem, které udává dimenzi výstupních dat. Ačkoliv si Julia uchovává vícerozměrná pole ve sloupcové majoritě, tak se na vstupu očekávají data tak, že dvojice vstupů a výstupů jsou na jednom řádku.

Třída \texttt{MwlsObject} má 3 různé implementace, které se liší způsobem hledání sousedů v rozsahu. Struktura \texttt{MwlsNaiveObject} používá k hledání naivní iteraci přes celou množinu vstupních dat, \texttt{MwlsCllObject} používá cell linked list a \texttt{MwlsKdObject} používá k-d strom. Pro snadnější vytvoření jednotlivých objektů jsou připraveny pomocné funkce \texttt{mwlsNaive}, \texttt{mwlsCll} a \texttt{mwlsKd}.

V teoretické části se pro aproximaci používala (téměř) libovolná sada funkcí $f_1, \ldots, f_l$. V implementaci se však pouze předá celočíselný parametr \texttt{maxDegree}, který určuje maximální stupeň \textit{polynomiální báze} \texttt{b} použité k aproximaci v každé z jednotlivých dimenzí. Demonstrujeme na příkladu \ref{exer:basis-example}.
\begin{priklad}
  \label{exer:basis-example}
  Pro parametr $\texttt{maxDegree} = 2$ a dimenzi výstupních dat $m = 2$ se vytvoří sada funkcí $f_1(x_1, x_2), \ldots, f_{12}(x_1, x_2)$
  \begin{multline}
    f_1 = (1, 0), f_2 = (x, 0), f_3 = (y, 0), f_4 = (x^2, 0), f_5 = (xy, 0), f_6 = (y^2, 0),\\
    f_7 = (0, 1), f_8 = (0, x), f_9 = (0, y), f_{10} = (0, x^2), f_{11} = (0, xy), f_{12} = (0, y^2)\\
  \end{multline}
  která bude použita pro konstrukci aproximace.
\end{priklad}
Lze si však všimnout, že se mnoho funkcí často opakuje a liší se pouze ve složce, která je nenulová. Proto budeme generovat bázi tak, aby se funkce neduplikovaly. Výsledkem příkladu \ref{exer:basis-example} tedy bude pouze $f_1(x, y) = 1, f_2(x, y) = x, f_3(x,y) = y, f_4(x,y)=x^2,f_5=xy,f_6=y^2$.

Přejděme k implementaci výpočtu matice $\mx{A}$ a vektoru $\vec{b}$ z rovnice \ref{eq:sle-mwls}. Jak již bylo naznačeno, mnoho funkcí se v polynomiální bázi použité k aproximaci opakuje. Proto se místo vektoru $\vec{b}$ použije matice $\mx{B}$, kde v řádcích jsou jednotlivé výstupní vektory $y_i$.



\begin{conclusion}
  %sem napište závěr Vaší práce
\end{conclusion}

\printbibliography

\appendix

\chapter{Obsah přiloženého CD}

\begin{figure}
  \dirtree{%
    .1 readme.txt\DTcomment{stručný popis obsahu CD}.
    .1 exe\DTcomment{adresář se spustitelnou formou implementace}.
    .1 src.
    .2 impl\DTcomment{zdrojové kódy implementace}.
    .2 thesis\DTcomment{zdrojová forma práce ve formátu \LaTeX{}}.
    .1 text\DTcomment{text práce}.
    .2 thesis.pdf\DTcomment{text práce ve formátu PDF}.
    .2 thesis.ps\DTcomment{text práce ve formátu PS}.
  }
\end{figure}

\end{document}
